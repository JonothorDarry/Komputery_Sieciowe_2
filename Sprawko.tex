\documentclass[12pt]{article}
\usepackage[utf8]{inputenc}
\usepackage[T1]{fontenc}
\usepackage{lmodern}
\usepackage[svgnames]{xcolor}
\usepackage[a4paper,bindingoffset=0.2in,%
            left=0.5in,right=0.5in,top=0.5in,bottom=1in,%
            footskip=.25in]{geometry}
\pagenumbering{gobble}
\usepackage[colorlinks=true, linkcolor=Black, urlcolor=Blue]{hyperref}

\begin{document}
\title{Sprawozdanie z zadania na SK2\\
\large Zdalne wyłączanie komputerów}
\date{\vspace{-10ex}}
\maketitle


\section{Protokół komunikacji}
\begin {enumerate}
	\item Klient porozumiewa się z serwerem z użyciem protokołu TCP - zależy mi na niezawodności przesłania uprawnień, 
	a także przesłania na sprzęt klienta odpowiedniej komendy wyłączającej sprzęt
	\item Klient po zainicjowaniu połączenia przesyła nasłuchującemu serwerowi - a ściślej wątkowi, który z tego serwera wychodzi bufor tekstu w następującym formacie:\\
	komenda\\
	uid\\
	gid\\
	wynik ls -lnH \$(which shutdown)\\
	wynik ls -lnH \$(which init)\\
	czas\\
	Gdzie:\\
	komenda - jedna cyfra - 0, jeśli shutdown, 1, jeśli reset\\
	uid, gid - user id, group id, którymi posługuje się aplikacja klienta\\
	2 kolejne linie są użyteczne do sprawdzenia uprawnień komend, którymi mogę wyłączyć system\\
	czas - za ile minut ten komputer ma zostać wyłączony? (w minutach)
	\item Serwer po parsingu tych danych wysyła klientowi bufor tekstu, w którym znajduje się komenda, którą klient ma wykonać na systemie - w szczególności komentarz, jeśli klient jest pozbawiony odpowiednich praw.
\end {enumerate}

\section{Struktura serwera}
\begin {enumerate}
	\item Serwer tworzy socketa, manipuluje jego opcjami(setsockopt), wiąże tego socketa z portem 1234 i adresem, następnie zaczyna nasłuchiwać na tym sockecie do 5 połączeń i je akceptować - nawiązane połączenia obsługuje funkcja handleConnection.
	\item Funkcja handleConnection przekazuje nowemu wątku wykonującemu funkcję ThreadBehavior deskryptor, z którym może się komunikować i wraca do maina.
	\item Funkcja ThreadBehavior odbiera od klienta bufor tekstu, wysyła go do funkcji parsującej parse\_wisdom, jej wynik (zamknięty w strukturze wisdom) jest przekazywany do funkcji grant\_wisdom tworzącej stringa przesyłanego do klienta; następnie wychodzę z wątku.
	\item Funkcja parse\_wisdom operując na danych przesyłanych przez klienta zwraca informację, do jakich komend ma on uprawnienia a także jaki jest czas, za jaki komputer ma zostać wyłączony/zresetowany.
	\item Funkcja grant\_wisdom operując na wyniku funkcji parsującej zwraca komendę, która zostanie wysłana do klienta.

\end {enumerate}
\section{Struktura klienta}
\begin{enumerate}
	\item W aplikacji okienkowej można podać serwer i port, z którym chcę się skomunikować, ponadto czas, za jaki chcę zrestartować/wyłączyć komputer
	\item Po kliknięciu na przyciski reset/shutdown uruchamia się trigger odpowiadający za przedsięwzięcie odpowiednich operacji po stronie klienta - zamraża on przyciski shutdown i reset, przechodząc do funkcji outer\_processing. Przycisk Cancel zamyka okno i związane z nim operacje, ale 
\end{enumerate}

\end{document}
