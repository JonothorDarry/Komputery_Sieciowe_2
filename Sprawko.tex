\documentclass[12pt]{article}
\usepackage[utf8]{inputenc}
\usepackage[T1]{fontenc}
\usepackage{lmodern}
\usepackage[svgnames]{xcolor}
\usepackage[a4paper,bindingoffset=0.2in,%
            left=0.5in,right=0.5in,top=0.5in,bottom=1in,%
            footskip=.25in]{geometry}
\pagenumbering{gobble}
\usepackage[colorlinks=true, linkcolor=Black, urlcolor=Blue]{hyperref}

\begin{document}
\title{Sprawozdanie z zadania na SK2\\
\large Zdalne wyłączanie komputerów\\
\large Sebastian Michoń 136770}
\date{\vspace{-10ex}}
\maketitle

\section{Protokół komunikacji}
\begin {enumerate}
	\item Klient zarządzający (client) i klient do wyłączenia (node) porozumiewają się z serwerem z użyciem protokołu TCP - zależy mi na niezawodności przesłania identyfikatora node'a i polecenia od clienta, 
	a także przesłania na sprzęt node'a odpowiedniej komendy wyłączającej sprzęt
	\item Client po zainicjowaniu połączenia przesyła nasłuchującemu serwerowi - a ściślej wątkowi, który z tego serwera wychodzi bufor tekstu w następującym formacie:\\
	komenda\\
	(czas) <lista identyfikatorów>\\
	Gdzie:\\
	komenda - jedna cyfra - 0, jeśli shutdown, 1, jeśli reset, 2, jeśli poszukiwanie identyfikatora.\\
	czas - tylko jeśli komenda to 0 albo 1 - czas, za jaki ma zostać wykonana komenda.\\
	<lista identyfikatorów> - jeśli komenda=2, lista identyfikatorów, które chcę zdobyć, żeby uzyskać je (możliwość wykonania na nich operacji shutdown i reset). Jeśli 0 lub 1 - lista identyfikatorów, na których ma zostać wykonane podane polecenie.
	
	\item Serwer po parsingu tych danych wysyła wszystkim node'om powiązanym z podanymi identyfikatorami bufor tekstu, w którym znajduje się komenda, którą klient ma wykonać na systemie.
	
	\item Wszystkie komunikaty mają długość 1024 znaków
\end {enumerate}

\section{Struktura serwera}
\begin {enumerate}
	\item Serwer tworzy socketa, manipuluje jego opcjami(setsockopt), wiąże tego socketa z portem 1234 i adresem, następnie zaczyna nasłuchiwać na tym sockecie do 20 połączeń i je akceptować - nawiązane połączenia obsługuje funkcja handleConnection.
	\item Funkcja handleConnection przekazuje nowemu wątku wykonującemu funkcję ThreadBehavior deskryptor, z którym może się komunikować i wraca do maina.
	\item Funkcja ThreadBehavior odbiera od klienta bufor tekstu: jeśli zaczyna się on od litery n - dane pochodzą od node'a, identyfikator jest dodawany na wolną pozycję w tablicy identyfikatorów, następnie blokuję wątek funkcją pthread\_cond\_wait - po wysłaniu sygnału na zmienną warunkową związaną z tym waitem node otrzymuje polecenie od serwera, które ma wykonać.
	 \item Jeśli bufor zaczyna się od cyfry 2, jest to zapytanie od klienta o identyfikatory - jeśli identyfikator znajduje się w tablicy identyfikatorów serwera, to jest on wysyłany z powrotem; jeśli nie, to nie.
	 \item Jeśli bufor zaczyna się od cyfry 0 albo 1, jest to zapytanie o wyłączenie określonego zbioru node'ów powiązanych z odpowiednimi identyfikatorami - serwer zapisuje odpowiednie polecenia do tablicy poleceń, a następnie wysyła sygnał na zmienną warunkową jeśli jest w stanie się połączyć z danym node'em (czyli np. nie został wyłączony razem z komputerem).
	 
\end {enumerate}
\section{Struktura klienta}
\begin{enumerate}
	\item W aplikacji okienkowej można podać serwer i port, z którym chcę się skomunikować, ponadto czas, za jaki chcę zrestartować/wyłączyć komputer i identyfikatory, które chcę uzyskać.
	\item Wpisanie nazw identyfikatorów oddzielonych spacjami przy podaniu adresu i portu serwera, a następnie kliknięciu przycisku "Add identificators" wysyła zapyanie do serwera o podane identyfikatory używając funkcji outer\_processing  - jeśli istnieją, trafiają na listę po lewej stronie; jeśli nie, są pomijane.
	\item Po kliknięciu na przyciski reset/shutdown uruchamia się trigger odpowiadający za przedsięwzięcie odpowiednich operacji po stronie node'a dla identyfikatorów zaznaczonych na liście identyfikatorów - zamraża on wszystkie przyciski poza Cancelem, przechodząc do funkcji outer\_processing. Przycisk Cancel zamyka okno i związane z nim operacje.
	\item funkcja outer\_processing wsadza dane od gui do struktury i wrzuca je do nowego wątku wykonującego funckję parse\_connection, wychodząc z funkcji bez pthread\_joina - dzięki temu aplikacja się nie wiesza.
	\item funkcja parse\_connection tworzy socketa, a następnie łączy się z podanym przez użytkownika adresem i portem. Dalej wywołuję funkcję zarządzającą połączeniem handleConnection i zamykam socketa.
	\item Funkcja handleConnection wysyła do serwera podane aplikacji przez użytkownika informacje i otrzymuje odpowiedź formatu:\\
	(2v3) <lista identyfikatorów>\\
	2 - Odpowiedź na szukanie identyfikatorów, lista jest listą znalezionych identyfikatorów\\
	3 - Odpowiedź na wyłączanie/reset node'ów, lista jest listą znalezionych niepoprawnych identyfikatorów - są one usuwane z listy po lewej.
	\item Timer odczytuje co 1 sekundę informację o tym, czy thread się zakończył - jeśli tak, to wykonywane jest polecenie wysłane przez serwer, odblokowywane są przyciski.
	\item Przycisk "remove identificators" pozwala usunąć wybrane na liście identyfikatory z listy. 
\end{enumerate}

\section{Struktura node'a}
\begin{enumerate}
	\item Po podaniu serwera, portu i identyfikatora sprawdzana jest nazwa identyfikatora; jeśli nie składa się wyłącznie ze znaków alfanumerycznych, proces się kończy.
	\item Następnie tworzony jest wątek i nawiązywana jest komunikacja z serwerem w taki sam sposób, jak w przypadku zwykłego klienta.
	\item Node wysyła serwerowi identyfikator, następnie oczekuje na komunikaty zwrotne - jeśli zaczyna się takowy od ! - error, jeśli nie od ! - polecenie, które node ma wykonać na swoim komputerze.
\end{enumerate}

\section{Uruchomienie}
\begin{enumerate}
	\item Uruchomienie serwera:\\
		cd ...(miejsce, gdzie zaszyty jest projekt)\\
		gcc ser.c -o ser -Wall -l pthread\\
		./ser
	\item Uruchomienie klienta:\\
		cd .../QtServerClientApp (w miejscu, gdzie jest projekt)\\
		qmake\\
		make\\
		./QtServerClientApp
	\item Uruchomienie node'a:\\
	cd ...(miejsce, gdzie zaszyty jest projekt)\\
	gcc node.c -o node -Wall -l pthread\\
	./node localhost 1234 Random\\
	\#Wiąże node'a z identyfikatorem o nazwie Random - od tego momentu wywołanie po stronie klienta w polu "Identificators to add" nazwy Random i kliknięcie przycisku "Add identificators" będzie skutkowało dodaniem tego identyfikatora do listy po lewej stronie.
		
\end{enumerate}

\end{document}
